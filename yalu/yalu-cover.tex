\documentclass{article} %%% yalu-cover.tex :: mito / nyamcoder %%%
\pagestyle{empty}
\usepackage[svgnames]{xcolor}
\usepackage{pst-plot,pst-text,pst-grad,pst-3dplot,pst-slpe}
\usepackage{hangul,graphicx}
%\usepackage{pifont} oder {bbding} funktionieren mit EUC-KR bzw. Λ nicht für Dingbat, ähnlich U+2740

\begin{document}
\makeatletter
\디나루 % = \fontfamily{dn}\selectfont (Dinaru-Font)
\color{orange}
\fontseries{bux}\selectfont  % fett + ultrabreiter Schnitt

\begin{pspicture}(0,0)(8,8)
%%% Morgenstimmungs-Image
\rput(-2,-12){\psframe[linewidth=0.01,fillstyle=slopes,slopecolors=0 .98 1 1 34 1 1 .9 50 1 .75 0 58 .5 1 .5 100 0 .5 .5 5,slopeangle=270,slopesteps=1000](0,0)(15.7,23)}

%%% Sonnen-Image mit Schriftvergrößerungalgorithmus + Text
\rput(9,7){\pscircle[linestyle=none,fillstyle=gradient,gradbegin=pink,gradend=yellow,gradmidpoint=0,gradangle=25](0,0){2.4}}
% "A \DoPerCharacter macro adapted from Juergen Schlegelmilch (posted on comp.text.tex - January 1998)"
%%% Code-Syntax an die aktualisierten pst-Pakete angepasst
\def\DoPerCharacter#1#2#3\@nil{#1#2 \edef\@tempa{#3}%
\ifx\@tempa\empty
\else
\DoPerCharacter#1#3\@nil
\fi}

\def\PerCharacter#1#2{\DoPerCharacter#1#2\@nil}
\pst@dimh=0.7pt % Höhenaddition pro Schritt

\def\CharacterAction#1{%
\psscalebox{1 \pst@number{\pst@dimh}}{#1} % psscalebox=Laufweite
\advance\pst@dimh by 0.15pt}

\rput(9.2,6.5){\pstextpath{\psplot[linestyle=none,polarplot=true,plotpoints=300,unit=5]{500}{-500}{1 2.7182818 x 200 div exp 1 add div}}% Spirale; Linienoption: linecolor=blue
{\PerCharacter{\CharacterAction}{{{\red{\quad  이 미 륵} }}Mirok{{ }}Li}} }

%%% Gebirgs-Image
\pspolygon[linestyle=none,fillstyle=slopes,slopecolors=0 1 1 1 5 1 1 .9 9 .75 .75 .75 50 .25 .25 .25 4,slope angle=270,slopesteps=500](-.5,-10)(2,1)(4.5,2)(5.5,1)(6.8,2.3)(8.5,1.2)(10,2)(11.4,1)(11.1,-10)
%%% --> slopecolors/letzte Zahl: Anzahl der zu berücksichtigenden Farbstufen

%%% Fluss-Image
\pst@dimh=2pt
\def\CharacterAction#1{%
\psscalebox{1.05 \pst@number{\pst@dimh}}{#1}
\advance\pst@dimh by 0.2pt}
\renewcommand\blue{\color{DarkSlateBlue}}
%%% orig.:
%\pstextpath{\pscurve[linestyle=none](-1,0)(2.5,-.8)(1.2,-2.6)(10,-4.2)(11.5,-5)(12.5,-6)}{\blue\PerCharacter{\CharacterAction}{ {{압}}{{록}}{{강}}{{은}}{{ }}{{흐}}{{른}}{{다}}{{ -- }}DER{{ }}YALU{{ }}FLIESST}}
% Linienoption: [linecolor=orange,linewidth=2pt]
%%% neu für manuelle Korrektur:
\pstextpath{\pscurve[linestyle=none](-1,-2)(2.5,-2.8)(1.2,-4.6)(10,-6.2)(11.5,-7)(12.5,-8)}{\blue\PerCharacter{\CharacterAction}{{{압}}{{록}}{{강}}{{은}}{{ }}{{ }}{{ }}{{ }}{{른}}{{다}}{{ -- }}DER{{ }}YALU{{ }}FLIESST}}
%%% manuelle Korrektur:
\rput(1.9,-3.95){\resizebox{.85cm}{1.3cm}{\rotatebox{12}{\blue 흐}}}

%%% diverse Untertitel
\rput(6.6,-3.4){\pstilt{95}{\psscalebox{.77 2}{\Large\color{Gold}  한국에서의 소년시대}}}
\rput(6.55,-4.2){\pstilt{80}{\psscalebox{.8 2}{\Large\color{Orange} Eine Jugend in Korea}}}
\rput(5.6,-7.8){\pstilt{80}{\psscalebox{.4 1}{\color{MediumAquamarine} 한국어· 독일어판 \quad
\raisebox{-.35ex}{\large\color{CadetBlue} Koreanisch-Deutsche Ausgabe}}}}

% Logo
\rput(5.5,-9.3){\includegraphics[bb=148 642 175 668,width=4mm]{./mugunghwa.ps}\raisebox{8.2mm}{\,\textcolor{blue}{$\mu ugunghwa$}}}

% Hilfsgitter
%\psgrid[gridcolor=red,gridlabelcolor=blue,subgriddiv=10](-3,-13)(14,12)
\end{pspicture}
\end{document}


* Kompilation

$ lambda yalu-cover && dvips yalu-cover && ps2pdf yalu-cover.ps
