%%% toc-ko2.tex :: mito / nyamcoder %%%
\documentclass[a4paper,11pt]{book}%  vgl. memoir, oblivoir, article, scrbook ... 
\usepackage{boxedminipage}
\usepackage[hangul]{kotex}
\usepackage{times}
\usepackage{titletoc}
\contentsmargin{0pt}	% Abstand zum re. Rand 
\titlecontents{chapter}[0em]	% einzug links, nicht optional! 
	{\addvspace{1.4pc}\bfseries}
	{{\Huge\thecontentspage\quad}}{}{}  
%	{\contentspush{\Huge\thecontentspage\quad}}{}{} 	

%  anm: {\thecontentslabel\ } ==> 제1장 ; bsp: lb2-67 
\newcommand\xquad
	{\hspace{1.4em plus .2em minus .2em}}	
\titlecontents*{section}[0pt]
	{\small}{}{}
	{,~\thecontentspage}[\xquad\textbullet\xquad][.]
\titlecontents*{subsection}[0pt]{\footnotesize}{}{}{}[\quad//\quad][\quad/\quad][]  % zu beginn " // ", id mitte " / ", am schluss " " ; 	% dazu +level bei subsection
	
%%% nach lbc2-66f 
\setcounter{tocdepth}{2}% erhöht wg subsection 
	
%\setlength\textheight{27cm}
%\setlength\voffset{-2cm}
%\setlength\textwidth{16cm}

\begin{document}	% toc: 48ff 
\begin{boxedminipage}[t]{\linewidth}

\tableofcontents		% muss sein, wenn noch kein .toc vorhanden 
%%%% toc-ko2.tex :: mito / nyamcoder %%%
\documentclass[a4paper,11pt]{book}%  vgl. memoir, oblivoir, article, scrbook ... 
\usepackage{boxedminipage}
\usepackage[hangul]{kotex}
\usepackage{times}
\usepackage{titletoc}
\contentsmargin{0pt}	% Abstand zum re. Rand 
\titlecontents{chapter}[0em]	% einzug links, nicht optional! 
	{\addvspace{1.4pc}\bfseries}
	{{\Huge\thecontentspage\quad}}{}{}  
%	{\contentspush{\Huge\thecontentspage\quad}}{}{} 	

%  anm: {\thecontentslabel\ } ==> 제1장 ; bsp: lb2-67 
\newcommand\xquad
	{\hspace{1.4em plus .2em minus .2em}}	
\titlecontents*{section}[0pt]
	{\small}{}{}
	{,~\thecontentspage}[\xquad\textbullet\xquad][.]
\titlecontents*{subsection}[0pt]{\footnotesize}{}{}{}[\quad//\quad][\quad/\quad][]  % zu beginn " // ", id mitte " / ", am schluss " " ; 	% dazu +level bei subsection
	
%%% nach lbc2-66f 
\setcounter{tocdepth}{2}% erhöht wg subsection 
	
%\setlength\textheight{27cm}
%\setlength\voffset{-2cm}
%\setlength\textwidth{16cm}

\begin{document}	% toc: 48ff 
\begin{boxedminipage}[t]{\linewidth}

\tableofcontents		% muss sein, wenn noch kein .toc vorhanden 
%%%% toc-ko2.tex :: mito / nyamcoder %%%
\documentclass[a4paper,11pt]{book}%  vgl. memoir, oblivoir, article, scrbook ... 
\usepackage{boxedminipage}
\usepackage[hangul]{kotex}
\usepackage{times}
\usepackage{titletoc}
\contentsmargin{0pt}	% Abstand zum re. Rand 
\titlecontents{chapter}[0em]	% einzug links, nicht optional! 
	{\addvspace{1.4pc}\bfseries}
	{{\Huge\thecontentspage\quad}}{}{}  
%	{\contentspush{\Huge\thecontentspage\quad}}{}{} 	

%  anm: {\thecontentslabel\ } ==> 제1장 ; bsp: lb2-67 
\newcommand\xquad
	{\hspace{1.4em plus .2em minus .2em}}	
\titlecontents*{section}[0pt]
	{\small}{}{}
	{,~\thecontentspage}[\xquad\textbullet\xquad][.]
\titlecontents*{subsection}[0pt]{\footnotesize}{}{}{}[\quad//\quad][\quad/\quad][]  % zu beginn " // ", id mitte " / ", am schluss " " ; 	% dazu +level bei subsection
	
%%% nach lbc2-66f 
\setcounter{tocdepth}{2}% erhöht wg subsection 
	
%\setlength\textheight{27cm}
%\setlength\voffset{-2cm}
%\setlength\textwidth{16cm}

\begin{document}	% toc: 48ff 
\begin{boxedminipage}[t]{\linewidth}

\tableofcontents		% muss sein, wenn noch kein .toc vorhanden 
%%%% toc-ko2.tex :: mito / nyamcoder %%%
\documentclass[a4paper,11pt]{book}%  vgl. memoir, oblivoir, article, scrbook ... 
\usepackage{boxedminipage}
\usepackage[hangul]{kotex}
\usepackage{times}
\usepackage{titletoc}
\contentsmargin{0pt}	% Abstand zum re. Rand 
\titlecontents{chapter}[0em]	% einzug links, nicht optional! 
	{\addvspace{1.4pc}\bfseries}
	{{\Huge\thecontentspage\quad}}{}{}  
%	{\contentspush{\Huge\thecontentspage\quad}}{}{} 	

%  anm: {\thecontentslabel\ } ==> 제1장 ; bsp: lb2-67 
\newcommand\xquad
	{\hspace{1.4em plus .2em minus .2em}}	
\titlecontents*{section}[0pt]
	{\small}{}{}
	{,~\thecontentspage}[\xquad\textbullet\xquad][.]
\titlecontents*{subsection}[0pt]{\footnotesize}{}{}{}[\quad//\quad][\quad/\quad][]  % zu beginn " // ", id mitte " / ", am schluss " " ; 	% dazu +level bei subsection
	
%%% nach lbc2-66f 
\setcounter{tocdepth}{2}% erhöht wg subsection 
	
%\setlength\textheight{27cm}
%\setlength\voffset{-2cm}
%\setlength\textwidth{16cm}

\begin{document}	% toc: 48ff 
\begin{boxedminipage}[t]{\linewidth}

\tableofcontents		% muss sein, wenn noch kein .toc vorhanden 
%\input{toc-ko2.toc}\contentsfinish
\end{boxedminipage}

\chapter*{머리말}
\addcontentsline{toc}{chapter}{\protect\numberline{}머리말} % nötig, wenn \chapter* uä im toc sein sollen  
한국은 광대한 문화 유산중에서 국보와 보물에 따라 분류된 작품이 많은데\,\cite{k1} 그 중에서 유네스코 세계유산에 문화재로 지정돤 것도 있다.\cite{k2}

\chapter{남한의 국보}	% 1 ebene herausgenommen  
\section{숭례문 (1호)}
또는 남대문
\section{불국사}
\subsection{다보탑 (20호)}
\subsection{석가탑 (21호)}
\subsection{연화교\,·\,칠보교 (22호)}
\subsection{청운교\,·\,백운교 (23호)}
\subsection{금동비로자나불좌상 (26호)}
\subsection{금동아미타여래좌상 (27호)}
%\section{석굴암 (24호)}
%\section{시각 예술\,·\,공예의 작품}
%\section{금동미륵보살반가상 (83호)}
%%%% \section{미술품\,·\,문서\,·\,인쇄물}
\section{해인사 대장경판 (32호)}
\section{훈민정음 (70호)}
%\section{천마도 (207호)}
\chapter{남한의 보물}
보물\cite{k3}
\section{서울 화계사 동종 (제11-5호)}
\section{불국사 사리탑 (61호)}
\section{칠곡 송림사전탑 사리장엄구 (325호)}
\section{금동자물쇄 및 문고리 (777호)}
\section{용비어천가 (1483호)}
\chapter{북한의 국보}
\section{ 평양성}
\subsection{보통문 (3호)}
\subsection{대동문 (4호)}
\section{동명왕릉 (36호)}
\section{금강산}
\subsection{묘길상 (102호)}
\section{안악3호분 (67호)}
고구려 고분군\cite{k4}
\section{연복사종 (136호)}
\section{관음사대리석관음보살상 (156호)}
\chapter{우리 민족 국제의 문화재}
\section{고구려 고분군}
고구려 고분군\cite{k4}

\renewcommand\bibname{웹사이트}
\begin{thebibliography}{9}

\addcontentsline{toc}{chapter}{\protect\numberline{}\bibname}

\bibitem{k1} \verb#http://www.heritage.go.kr#, \enskip\verb#http://www.cha.go.kr#
\bibitem{k2} \verb#http://www.memorykorea.go.kr#,  「국보」를 참고
\bibitem{k3} \verb#http://ko.wikipedia.org/wiki/보물#  
\bibitem{k4} \verb#http://whc.unesco.org/en/list/1091#
\end{thebibliography}

\end{document}


* kompilation:

$ pdflatex toc-ko2 && pdflatex toc-ko2


* scrbook: headline sf, sn rechts unten, breiterer satzspiegel als book, kt o+außen+it, sn u+außen
* book: 	   "	  rm, sn mittig, kt o (kopfzeile), mit sn außen kap innen
* titletoc: lb2-62ff 
\contentsfinish
\end{boxedminipage}

\chapter*{머리말}
\addcontentsline{toc}{chapter}{\protect\numberline{}머리말} % nötig, wenn \chapter* uä im toc sein sollen  
한국은 광대한 문화 유산중에서 국보와 보물에 따라 분류된 작품이 많은데\,\cite{k1} 그 중에서 유네스코 세계유산에 문화재로 지정돤 것도 있다.\cite{k2}

\chapter{남한의 국보}	% 1 ebene herausgenommen  
\section{숭례문 (1호)}
또는 남대문
\section{불국사}
\subsection{다보탑 (20호)}
\subsection{석가탑 (21호)}
\subsection{연화교\,·\,칠보교 (22호)}
\subsection{청운교\,·\,백운교 (23호)}
\subsection{금동비로자나불좌상 (26호)}
\subsection{금동아미타여래좌상 (27호)}
%\section{석굴암 (24호)}
%\section{시각 예술\,·\,공예의 작품}
%\section{금동미륵보살반가상 (83호)}
%%%% \section{미술품\,·\,문서\,·\,인쇄물}
\section{해인사 대장경판 (32호)}
\section{훈민정음 (70호)}
%\section{천마도 (207호)}
\chapter{남한의 보물}
보물\cite{k3}
\section{서울 화계사 동종 (제11-5호)}
\section{불국사 사리탑 (61호)}
\section{칠곡 송림사전탑 사리장엄구 (325호)}
\section{금동자물쇄 및 문고리 (777호)}
\section{용비어천가 (1483호)}
\chapter{북한의 국보}
\section{ 평양성}
\subsection{보통문 (3호)}
\subsection{대동문 (4호)}
\section{동명왕릉 (36호)}
\section{금강산}
\subsection{묘길상 (102호)}
\section{안악3호분 (67호)}
고구려 고분군\cite{k4}
\section{연복사종 (136호)}
\section{관음사대리석관음보살상 (156호)}
\chapter{우리 민족 국제의 문화재}
\section{고구려 고분군}
고구려 고분군\cite{k4}

\renewcommand\bibname{웹사이트}
\begin{thebibliography}{9}

\addcontentsline{toc}{chapter}{\protect\numberline{}\bibname}

\bibitem{k1} \verb#http://www.heritage.go.kr#, \enskip\verb#http://www.cha.go.kr#
\bibitem{k2} \verb#http://www.memorykorea.go.kr#,  「국보」를 참고
\bibitem{k3} \verb#http://ko.wikipedia.org/wiki/보물#  
\bibitem{k4} \verb#http://whc.unesco.org/en/list/1091#
\end{thebibliography}

\end{document}


* kompilation:

$ pdflatex toc-ko2 && pdflatex toc-ko2


* scrbook: headline sf, sn rechts unten, breiterer satzspiegel als book, kt o+außen+it, sn u+außen
* book: 	   "	  rm, sn mittig, kt o (kopfzeile), mit sn außen kap innen
* titletoc: lb2-62ff 
\contentsfinish
\end{boxedminipage}

\chapter*{머리말}
\addcontentsline{toc}{chapter}{\protect\numberline{}머리말} % nötig, wenn \chapter* uä im toc sein sollen  
한국은 광대한 문화 유산중에서 국보와 보물에 따라 분류된 작품이 많은데\,\cite{k1} 그 중에서 유네스코 세계유산에 문화재로 지정돤 것도 있다.\cite{k2}

\chapter{남한의 국보}	% 1 ebene herausgenommen  
\section{숭례문 (1호)}
또는 남대문
\section{불국사}
\subsection{다보탑 (20호)}
\subsection{석가탑 (21호)}
\subsection{연화교\,·\,칠보교 (22호)}
\subsection{청운교\,·\,백운교 (23호)}
\subsection{금동비로자나불좌상 (26호)}
\subsection{금동아미타여래좌상 (27호)}
%\section{석굴암 (24호)}
%\section{시각 예술\,·\,공예의 작품}
%\section{금동미륵보살반가상 (83호)}
%%%% \section{미술품\,·\,문서\,·\,인쇄물}
\section{해인사 대장경판 (32호)}
\section{훈민정음 (70호)}
%\section{천마도 (207호)}
\chapter{남한의 보물}
보물\cite{k3}
\section{서울 화계사 동종 (제11-5호)}
\section{불국사 사리탑 (61호)}
\section{칠곡 송림사전탑 사리장엄구 (325호)}
\section{금동자물쇄 및 문고리 (777호)}
\section{용비어천가 (1483호)}
\chapter{북한의 국보}
\section{ 평양성}
\subsection{보통문 (3호)}
\subsection{대동문 (4호)}
\section{동명왕릉 (36호)}
\section{금강산}
\subsection{묘길상 (102호)}
\section{안악3호분 (67호)}
고구려 고분군\cite{k4}
\section{연복사종 (136호)}
\section{관음사대리석관음보살상 (156호)}
\chapter{우리 민족 국제의 문화재}
\section{고구려 고분군}
고구려 고분군\cite{k4}

\renewcommand\bibname{웹사이트}
\begin{thebibliography}{9}

\addcontentsline{toc}{chapter}{\protect\numberline{}\bibname}

\bibitem{k1} \verb#http://www.heritage.go.kr#, \enskip\verb#http://www.cha.go.kr#
\bibitem{k2} \verb#http://www.memorykorea.go.kr#,  「국보」를 참고
\bibitem{k3} \verb#http://ko.wikipedia.org/wiki/보물#  
\bibitem{k4} \verb#http://whc.unesco.org/en/list/1091#
\end{thebibliography}

\end{document}


* kompilation:

$ pdflatex toc-ko2 && pdflatex toc-ko2


* scrbook: headline sf, sn rechts unten, breiterer satzspiegel als book, kt o+außen+it, sn u+außen
* book: 	   "	  rm, sn mittig, kt o (kopfzeile), mit sn außen kap innen
* titletoc: lb2-62ff 
\contentsfinish
\end{boxedminipage}

\chapter*{머리말}
\addcontentsline{toc}{chapter}{\protect\numberline{}머리말} % nötig, wenn \chapter* uä im toc sein sollen  
한국은 광대한 문화 유산중에서 국보와 보물에 따라 분류된 작품이 많은데\,\cite{k1} 그 중에서 유네스코 세계유산에 문화재로 지정돤 것도 있다.\cite{k2}

\chapter{남한의 국보}	% 1 ebene herausgenommen  
\section{숭례문 (1호)}
또는 남대문
\section{불국사}
\subsection{다보탑 (20호)}
\subsection{석가탑 (21호)}
\subsection{연화교\,·\,칠보교 (22호)}
\subsection{청운교\,·\,백운교 (23호)}
\subsection{금동비로자나불좌상 (26호)}
\subsection{금동아미타여래좌상 (27호)}
%\section{석굴암 (24호)}
%\section{시각 예술\,·\,공예의 작품}
%\section{금동미륵보살반가상 (83호)}
%%%% \section{미술품\,·\,문서\,·\,인쇄물}
\section{해인사 대장경판 (32호)}
\section{훈민정음 (70호)}
%\section{천마도 (207호)}
\chapter{남한의 보물}
보물\cite{k3}
\section{서울 화계사 동종 (제11-5호)}
\section{불국사 사리탑 (61호)}
\section{칠곡 송림사전탑 사리장엄구 (325호)}
\section{금동자물쇄 및 문고리 (777호)}
\section{용비어천가 (1483호)}
\chapter{북한의 국보}
\section{ 평양성}
\subsection{보통문 (3호)}
\subsection{대동문 (4호)}
\section{동명왕릉 (36호)}
\section{금강산}
\subsection{묘길상 (102호)}
\section{안악3호분 (67호)}
고구려 고분군\cite{k4}
\section{연복사종 (136호)}
\section{관음사대리석관음보살상 (156호)}
\chapter{우리 민족 국제의 문화재}
\section{고구려 고분군}
고구려 고분군\cite{k4}

\renewcommand\bibname{웹사이트}
\begin{thebibliography}{9}

\addcontentsline{toc}{chapter}{\protect\numberline{}\bibname}

\bibitem{k1} \verb#http://www.heritage.go.kr#, \enskip\verb#http://www.cha.go.kr#
\bibitem{k2} \verb#http://www.memorykorea.go.kr#,  「국보」를 참고
\bibitem{k3} \verb#http://ko.wikipedia.org/wiki/보물#  
\bibitem{k4} \verb#http://whc.unesco.org/en/list/1091#
\end{thebibliography}

\end{document}


* kompilation:

$ pdflatex toc-ko2 && pdflatex toc-ko2


* scrbook: headline sf, sn rechts unten, breiterer satzspiegel als book, kt o+außen+it, sn u+außen
* book: 	   "	  rm, sn mittig, kt o (kopfzeile), mit sn außen kap innen
* titletoc: lb2-62ff 
