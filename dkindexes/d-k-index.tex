\documentclass[a4paper,11pt]{scrartcl}		% d-k-index.tex :: mito(XD) / nyamcoder
\usepackage[ngerman]{babel}	% comment out for creating a Koreanized document (as in screenshot ''B'')
\usepackage{kotex}		% screenshot ''C''
% \usepackage[hangul]{kotex}	% comment out the preceeding line and use this instead for a Koreanized document, ''B''
\usepackage[encapsulated]{CJK}
\newcommand{\cn}[1]{\begin{CJK}{UTF8}{gbsn}#1\end{CJK}}
\usepackage{makeidx}
\makeindex
\def\indexspace{\vskip 4em plus 2em minus 2em\relax}	% equalizing column heights
\setlength\textheight{19cm}
\usepackage{yfonts}
\usepackage{color}
\definecolor{haneul}{cmyk}{.2,0,0,0}
\newcommand{\bfcyan}[1]{청록색의 볼드체: \textbf{\textcolor{cyan}{#1}}}
\newcommand{\dasaek}{\makebox[0pt][r]{\textcolor{red}{\raisebox{.3ex}{$\longleftarrow$}}}여러 색의 글자\makebox[0pt][l]{\textcolor{red}{\raisebox{1.3ex}{\,$\nearrow$}}}\dotfill}

\begin{document}
\setlength{\parindent}{0pt}
\renewcommand\indexname{Stichwortverzeichnis}
%\part{Einführung}

\today\\[2cm]% p1
멍멍! -- wau-wau!
\index{感嘆詞@\fcolorbox{blue}{white}{感嘆詞·擬聲語·俗語}|(textit}
\index{Blah-fasel-tröt"!}  Klangfigur
\index{아@\underline{\ 아\ }|underline}
\index{아야}

\newpage% p2 
\index{아야|seealso{Aua"!}}
\index{blah}  klitschnass knochentrocken klirrend kalt  pardauz!  klappern
\index{아이고@\textcolor{cyan}{아이고}|textbf}
zack!\index{Blah-fasel-tröt"!} Pipapo  Schnickschnack
zack!\index{Blah-fasel-tröt"!} Paperlapapp!

\newpage% p3 
bumm!\index{Blah-fasel-tröt"!} rumms!
\index{blah!blubb} 꼬불꼬불 졸졸 흐르다 
\index{ㅠㅠㅠㅠㅠ} 펄럭펄럭  냠냠  부르릉
\index{갠소@{\Hugeᄀ\textcolor{cyan}{ᅢ}\textcolor{blue}{ᆫ}\textcolor{blue}{ᄉ}\textcolor{cyan}{ᅩ}\normalsize}|dasaek} 꿀꿀   삐약삐약    흐늘쩍흐늘쩍 
\index{ㅇ벗다} % wiki: ‘없다’의 오타/typo
\index{아|textsf}
\index{아!아!아@\textsf{아}}

\newpage% p4 
%\index{frohlocken@\textit{frohlocken}!Luja@\textswab{Luja"!}!\underline{\textswab{-- Luja, sag i"!"!}}}
\index{Aechz"!@\textswab{Ächz"!}!Inflektiv!\underline{\textswab{Erikativ}}}
땡!\index{ᅵᇸ @ㅋㅋㅋ|textit}
\index{ᅵᇸ   @{\quadᆿ}} % {ᅵᇸ   @{\quadᆿ}} {ᆿ@{\quadᆿ}}:  ᆿ comes first
\index{Blah-fasel-tröt"!}
\index{1337|see{l33t}}	% l33t

\newpage% p5 
%\index{Aetsch"!@Ätsch"!}
\index{Aua@\colorbox{cyan}{\textcolor{white}{\textbf{Aua"!}}}}
덜컹!\index{감우사|fbox}
%\index{感嘆詞}

\newpage% p6 
리트(l33t, leet, 1337)\index{리트@l33t{\,}(리트)} 

\newpage% p7 
\index{2B\textbar"!2B}	% "사느냐 죽느냐" ; "| --> --- ! 
%\index{1337}
\index{blubb@\textsf{blubb}|see{blah}}
앗!\index{아@\textbf{아}|textcolor{cyan}}
\index{아!아|colorbox{cyan}}
\index{*fg*}
\index{bbb|fcolorbox{blue}{haneul}}
%\index{ᄒᆞᆫ}
\index{ᅵᇸ@\quadᅵᇸ|bfcyan}	%	HY PUA U+(1175+11F8)
\index{ᅷᆽ@\quad\textcolor{cyan}{ᅷ}ᆽ}	%	HY PUA U+(1177+11BD)
%\index{ᅗᇱ}	%
\index{ᅡᆩ@ᄁᇟ}	%	HY PUA U+(1101+11DF)
\index{ᆅᇃ@\quadᆅ\textcolor{cyan}{ᇃ}}	% ᆅᇃ	HY PUA U+(1185+11C3)
\index{ᅵᇸ  @ᄏ}	%	HY PUA U+(110F)
%\begin{CJK}{UTF8}{}	%%% mj (o)utbt don't work --> metafont!! 
%\fontencoding{T1}\selectfont\CJKchar{"011}{"00F}
%\end{CJK}
 
%\index{아!아!啞@{\sf 啞}}
\index{아@\underline{\ 아}}
\index{感嘆詞@\fcolorbox{blue}{white}{感嘆詞·擬聲語·俗語}|)textit}
발성법(發聲法)/Phonation
감우사(感嘆詞)/Interjektion
의성어(擬聲語)/Onomatopoesie
의태어(擬態語)/Lautmalerei
속어(俗語)/Slang
게시판 언어(揭示板言語)/Internet-, Forumslang/Netzjargon 채팅/Chatten
\index{캐}
\newpage

\setcounter{page}{18}
헉!
\index{아!아!\cn{啊, 阿}\,\textsf{[中]}}
\index{아!아!「ああ"!」\,\textsf{[日]}}

\printindex

\end{document}



* compilation (1) for default index formatting (screenshot ''A''):

$ latex d-k-index.tex
$ makeindex d-k-index
$ latex d-k-index.tex
$ dvipdfmx d-k-index.dvi


* compilation (2) for Koreanized version w/ standard kotex.ist style (screenshot ''B''):

$ latex d-k-index.tex
$ komkindex.pl -s kotex d-k-index
$ latex d-k-index.tex
$ dvipdfmx d-k-index.dvi


* compilation (3) for German version w/ enhanced sorting macro, individual  kotex-de.ist style and Hangul before Latin (screenshot ''C''):

$ latex d-k-index.tex
$ (/path/to/)komkindex-de.pl  -s (/path/o/)kotex-ga-de.ist  -k  d-k-index
$ latex d-k-index.tex
$ dvipdfmx d-k-index.dvi


* notes:

a)  new files need a texhash for becoming registered by the TeX system
b)  pdflatex may used instead of latex && dvipdfmx
c)  optionally rename the final different PDFs
d)  the provided komkindex-de.pl itself doesn't work; the original komkindex.pl by D-H Kim needs to be edited and renamed accordingly
e)  when playing with index files, back them up to prevent overwriting (like the .ind)
