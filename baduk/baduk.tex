\documentclass{article}	%% baduk.tex :: mito / nyamcoder %%

\usepackage{fontspec}
%\font\kg="HY백송B" at 14pt		% ttf; needs HWP installed -- or choose similar a font
%\font\kf="HY백송B" at 10pt		%           -- dito --
%\font\kb="은 자모 바탕" at 10pt		% ttf; Un font collection, download from here: http://kldp.net/projects/unfonts/download
						% after unzipping and copying to your font folder, in Linux don't forget to run ``fc-cache -fv''

\usepackage{igo}				% not included in TL -- needs extra installation
\igofontsize{10}

\begin{document}
\centering
%\shortstack{

%\hspace{.28mm}
\flushleft\hspace*{1.97cm}
\begin{minipage}{3.5cm}
\black{a7,a8,b2,b8,c2,c8,d2,d3,d4,d5,d6,d7,d8}
\white{a3,a6,b3,b5,c3,c4}
\showgoban[a1,f10]
\end{minipage}
\begin{minipage}{3.5cm}

\end{minipage}
%}
\\[1em]
%\shortstack{
\centering

\black[1]{c5}
\showgoban[a1,f10]\cleargobansymbols\quad
\white[2]{a4}
\showgoban[a1,f10]\cleargobansymbols\quad
\black[3]{b6}\white[\igocross]{a6}
\showgoban[a1,f10]%\cleargobansymbols
%}
\end{document}

%%% igo: lgc690ff

NOTE: The igo-package is not part of any LaTeX disribution. Download it from
[1]  http://mirror.ctan.org/fonts/igo.zip
or
[2]  http://faq.ktug.org/faq/Karnes/2006-05?action=download&value=igo-type1-1.zip

A: installing igo-stuff into the TeX-system (as a user)

0. identify $TEXMFHOME
$ kpsewhich -expand-var "\$TEXMFHOME" $HOME/texmf


1. creation of necessary folders according to the TDS:

$ mkdir -p ~/texmf/doc/latex/igo && mkdir -p ~/texmf/fonts/source/igo && mkdir -p ~/texmf/tex/latex/igo


2. unzip and copy the contents of [1] or [2] to -->

igo.sty --> ~/texmf/tex/latex/igo
igo/fonts --> ~/texmf/fonts/source/igo
other -->  ~/texmf/doc/latex/igo


3. introducing all igo-stuff to the TeX-system (as a user):

$ texhash (~/texmf)

(check with: $ kpsewhich igo.sty)

--> In case of using the download from [1], MetaFont will be activated to create vritual .pk-fonts at the initial source file compilation (B); this might take a while.


4. In case of downloading from [2] (recommended), the pre-compiled .pfb-PostScript fonts (Type1) will be used instead. However, this is needed to be done before:
$ updmap --enable Map=(/path/to/)igo.map



B: compiling the main document:

$ xelatex baduk(.tex)



~~~~~~~~~~~~~~~~~~~~~~~~~~~~~~~~~~~~~~~~~~~~~~~~~~~~~~

*** find other badul/igo-examples here: http://faq.ktug.org/faq/IgoPackage#s-3

